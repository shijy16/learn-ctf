\chapter{逆向工程}
\section{乙组基准-simple\_vm}
一个简单的vm,用switch语句做的,进去F5就能看到,推导出有哪些指令和格式就好了。题目也给了一段指令,要求输入三个数字,然后运行这段指令,对这三个数进行运算后,栈中某个位置结果是0,输入的三个数字就是flag。

最主要的点在于看栈、寄存器和其他参数的内存位置,然后把题目给的指令解析出来,列出算式解方程。

\section{乙组基准-anti\_patience}
这个题目直接用ida逆向会有一些地方解析失败,需要手动patch,把对应位置patch为nop,这里用LazyIDA插件,填充为nop后就可以生成伪代码了。

题目还用ptrace判断当前进程有没有被gdb调试,那个位置也需要patch,否则调试的时候随机数种子和正常运行的时候不一样。最后发现整个程序需要输入一段字符串,这段字符串进行很长一段逐字符运算后,得到一个res,然后题目中也有一个准备好的字符串,这个字符串逐字节和随机数进行运算,得到一个target\_res。最后对这两个结果进行比较,相等则输出flag,这个flag也是由随机数生成的。

解题步骤:
\begin{itemize}
    \item patch解析失败的地方。
    \item patch反调试的地方,使调试时随机种子不变。
    \item gdb调试,在检查结果的时候下断点,获取target\_res。
    \item 把输入字符串的运算过程copy出来,改成一个暴力求解过程。就可以获得flag了。
\end{itemize}