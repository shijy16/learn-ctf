\chapter{web入门}
web类题目是CTF比赛的主要题目,和二进制、逆向题目相比,不需要掌握底层知识。本章介绍web题目最常见的三类漏洞。


\section{信息收集}
信息搜集涵盖的面很广泛,包含备份文件、目录信息、Banner信息等,信息搜集主要依赖经验。


\subsection{敏感目录泄露}
通过敏感目录泄露通常可以获取网站的源代码和敏感的URL地址。
\subsubsection*{git泄露}

\paragraph*{常规git泄露}
直接用现成工具或脚本获取网站源码或flag。如:在确保目标URL含有.git的情况下,可以直接使用\href{https://github.com/denny0223/scrabble}{scrabble}来获取网站源码。命令如下:
\begin{lstlisting}
    ./scrabble http://example.com/
\end{lstlisting}

\paragraph*{git回滚}
利用scrabble获取网站源码后,部分情况下,flag会在之前的commit中被删除/修改,这时需要回滚。
\begin{lstlisting}
    git reset --hard HEAD^ #回滚到上一版本
    git log -stat #查看每个commit修改了什么
    git diff HEAD commit-id #查看当前版本与目标版本的区别
\end{lstlisting}

\paragraph*{git分支}
有时候flag不在默认分支中,需要切换其他分支,但大部分现成git泄露工具不支持分支,还原其他分支代码需要手工进行文件提取。功能较强的工具有\href{https://github.com/WangYihang/GitHacker}{GitHacker},用法:
\begin{lstlisting}
    python GitHacker.py [Website]
\end{lstlisting}
而后使用$git\ reflog$命令查看checkout记录,可以发现其他分支,然后修改/复用GitHacker代码来自动恢复分支。

\paragraph*{git泄露其他利用}
泄露的git中可能还有其他有用信息,比如说.git/config文件夹里面可能有access\_tocken信息,用来访问用户其他仓库。

\subsubsection*{SVN泄露}
SVN是源代码版本管理软件,管理员可能疏忽将SVN隐藏文件夹暴露在外。可以利用.svn/entries或wc.db获取服务器源码。

工具:https://github.com/kost/dvcs-ripper/,Seay-svn(windows)。


\subsubsection*{HG泄露}
HG会创建.hg隐藏文件记录代码、分支信息。

工具:https://github.com/kost/dvcs-ripper/

\subsubsection*{总结经验}

CTF线上赛往往有重定向问题,如访问.git后重定向,再访问.git/config后有内容返回,就有.git泄露问题。

目录扫描工具:https://github.com/maurosoria/dirsearch

\subsection{敏感备份文件}

\subsubsection*{gedit备份文件}
gedit:文件保存后会有一个后缀为'$\sim$'的文件。如$flag\sim$。

\subsubsection*{vim备份文件}
vim崩溃时会有一个$.swp$文件,可以用$vim\ -r$命令恢复。

\subsubsection*{常规文件}
\begin{itemize}
    \item robots.txt:CMS版本信息
    \item readme
    \item www.zip/rar/tar.gz: 常常是网站的备份源码
\end{itemize}


\subsection{Banner识别}
Banner:网站服务器对外显示的一些基础信息,如网站使用的框架等。可以据此尝试框架历史漏洞。

\subsubsection*{自行搜集指纹库}
github上有CMS指纹库,也有扫描器。

\subsubsection*{使用已有工具}
Wappalyzer工具:python工具,使用pip安装即可。

\subsubsection*{总结经验}
随意输入一些URL有时可以通过404或302跳转页面发现一些信息。


\section{SQL注入}
\section{任意文件读取}