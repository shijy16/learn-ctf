\chapter{逆向工程}
\section{静态分析}
\subsection{1-helloword}
打开直接F5查看伪代码得到flag:
\begin{lstlisting}
n1book{Welcome_to_reversing_world!}
\end{lstlisting}

\subsection{2-simpleCrackme}
直接查看伪代码:
\begin{lstlisting}[language=C]
int __cdecl main(int argc, const char **argv, const char **envp)
{
  size_t v3; // rbx
  int result; // eax
  char v5; // [rsp+Bh] [rbp-A5h]
  int i; // [rsp+Ch] [rbp-A4h]
  char v7[32]; // [rsp+10h] [rbp-A0h] BYREF
  char s[96]; // [rsp+30h] [rbp-80h] BYREF
  int v9; // [rsp+90h] [rbp-20h]
  unsigned __int64 v10; // [rsp+98h] [rbp-18h]

  v10 = __readfsqword(0x28u);
  strcpy(v7, "zpdt{Pxn_zxndl_tnf_ddzbff!}");
  memset(s, 0, sizeof(s));
  v9 = 0;
  printf("Input your answer: ");
  __isoc99_scanf("%s", s);
  v3 = strlen(s);
  if ( v3 == strlen(v7) )
  {
    for ( i = 0; i <= strlen(s); ++i )
    {
      if ( s[i] <= 96 || s[i] > 122 )
      {
        if ( s[i] <= 64 || s[i] > 90 )
          v5 = s[i];
        else
          v5 = (102 * (s[i] - 65) + 3) % 26 + 65;
      }
      else
      {
        v5 = (102 * (s[i] - 97) + 3) % 26 + 97;
      }
      if ( v5 != v7[i] )
      {
        puts("Wrong answer!");
        return 1;
      }
    }
    puts("Congratulations!");
    result = 0;
  }
  else
  {
    puts("Wrong input length!");
    result = 1;
  }
  return result;
}
\end{lstlisting}

首先有字符串:'zpdt\{Pxn\_zxndl\_tnf\_ddzbff!\}',需要构造一个输入字符串来满足要求,输入字符串首先会检查总长度是否和目标串一致,而后会逐字符经过一个运算后检查结果是否和给定字符串同样位置的字符一致。运算流程为:
\begin{enumerate}
    \item 若不是字母,则不变,直接比较。
    \item 若是大写字母,则$res = (102 \times (v - 65) + 3) \%26 + 65$。
    \item 若是小写字母,则$res = (102 \times (v - 97) + 3) \%26 + 97$。
\end{enumerate}
直接写个脚本暴力求解:
\begin{lstlisting}[language=python]
def crack(a):
    res = 0
    if a <= 96 or a > 122:
        if a <= 64 or a > 90:
            res = a
        else:
            res = (102 * (a - 65) + 3) % 26 + 65
    else:
        res = (102 * (a - 97) + 3) % 26 + 97;
    return res

if __name__ == '__main__':
    target = 'zpdt{Pxn_zxndl_tnf_ddzbff!}'
    ans = ''
    for i in range(len(target)):
        for j in range(0,250):
            if chr(crack(j)) == target[i]:
                ans += chr(j)
                break
    print(ans)
\end{lstlisting}
获得结果:chaf\{Hdi\_cdiaj\_fim\_aacbmm!\},输入目标程序后得到输出:'Congratulations!'。解题完成。


\subsection{2-simpleCrackme\_O3}
是上一题的经过编译器优化的版本,仅目标字符串解析出来有问题,需要手动设置为char[28]数组。

