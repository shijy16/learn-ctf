\chapter{逆向工程}
\section{逆向工程基础}
主要讲了一些汇编、工具的基本常识,略过。


\section{静态分析}
\subsection{IDA使用入门}
讲IDA怎么用。
数据类型快捷键:
\begin{itemize}
    \item U: 取消一个地方已有数据类型定义。
    \item D: 把一个地方变成数据,一直按会修改数据的长度。
    \item C: 让一个位置变成指令。
    \item A: 让一个位置为起点变成以$\backslash 0$结尾的ASCII字符串。
    \item *: 将一个位置定义为数组。
    \item O: 将一个位置定义为地址偏移。
\end{itemize}

函数操作快捷键:
\begin{itemize}
    \item 删除函数: 选中函数后DELETE。
    \item 定义函数: 选中行后P。
    \item 修改函数参数:函数窗口中选中函数Ctrl+E,反汇编窗口选中函数内部Alt+E。
\end{itemize}

导航操作快捷键:
\begin{itemize}
    \item 后退到上一位置:Esc。
    \item 前进到下一位置:Ctrl+Enter。
    \item 跳转到一个地址:G,然后输入地址/名称。
    \item 跳转到某一区段:Ctrl+S。
\end{itemize}

其他:IDA Python、字符串子窗口、十六进制子窗口。

\subsection{HexRays反编译器入门}
\begin{itemize}
    \item 生成伪代码:F5。
    \item Collapse declaration: 折叠函数。
    \item 修改标识符:在标识符上按N。
    \item 切换常量显示格式:右键。
    \item 修改标识符类型:Y。
    \item 添加结构体类型:Insert或右键,而后IDA会自动识别该类型。也可以添加头文件。
    \item 代码跳过:选中后右键,选择fill\ with\ nops。(需要装\href{https://github.com/L4ys/LazyIDA}{LazyIDA插件})
\end{itemize}

\subsection{IDA和HexRays进阶}
\begin{itemize}
    \item main函数查找: 找\_\_lib\_start\_main和start函数。
    \item FLIRT签名: 函数列表中底色为青色的函数表示因为签名问题识别失败,Shift+F5打开Signature列表,然后按Insert,自动新增签名库。
    \item HesRays函数分析失败:
    \begin{itemize}
        \item call\ analysis\ failed: 找函数调用参数时出错,需要手动修改函数的原型声明,如从'int\ \_\_thiscall'改为'int\ \_\_cdecl'。
        \item sp-analysis\ failed: 优化等级较高时,编译器省略了ebp使用,转而使用rsp引用局部变量,HexRays在跟踪rsp时出错。一般是由参数个数或调用约定出错导致IDA对栈指针变化量计算错误,可以在Option-General中打开Stackpointer分析错误。
        \item 指令分析错误: 有时候代码段中会有一些地方插入了不会到达的混淆,导致之后的指令解析错误,需要定位到错误点把这些地方用nop填充。
    \end{itemize}  
\end{itemize}

\section{动态调试和分析}
\subsection{OllyDBG和x64DBG调试}
都是windows平台下的调试器。
\subsubsection*{快捷键}
\begin{itemize}
    \item Ctrl+G: 跳转。
    \item F2: 设置/删除断点。
    \item F7: 单步步入。
    \item F8: 单步步过。
    \item F4: 运行至光标。
    \item F9: 运行。
    \item 读写断点:x64DBG中,"断点->硬件断点"或"读取或写入->选择长度"。
\end{itemize}

\subsubsection*{简单脱壳}
“壳”是一种特殊程序,在运行时对另一个程序进行变换,重新生成可执行文件。主要有加密壳和压缩壳。

UPX壳使用广泛,历史悠久,适用多种平台和架构。

脱壳方法:
\begin{itemize}
    \item 静态:UPX本身提供脱壳器,命令行参数-d即可,但有时因为版本不同会失败,可以用UPXShell方便地切换。
    \item 动态:系统预先加载目标程序时,会在栈和寄存器填充数据。壳程序运行前需要保存这些数据,而后要进行恢复。所以通常壳程序开头是pushad指令,这时候只需要在栈顶下硬件读取断点,就可以找到壳程序退出点,然后在退出点使插件进行脱壳。
    \begin{itemize}
        \item OllyDBG: 插件->OllyDump->脱壳正在调试的进程,选中获取EIP作为OEP。
        \item x64DBG: 插件->Scylla,IAT\ Autosearch, Get\ Imports,选中imports中有红叉的删除,单击Dump将内存转为可执行文件,单击Fix\ Dump修复导入表,完成修复后在IDA中加载。
    \end{itemize}
\end{itemize}
生成的程序能在IDA中分析,但无法运行,主要因为重定位信息没有修复。可以用CFF\ Explorer工具修改Nt\ Header的"Charatristics",勾选"Relocation info stripped from file",就可以阻止系统对该程序ASLR。

\subsection{GDB调试}
gdb命令:
\begin{itemize}
    \item r: 运行。
    \item c: 继续。
    \item si: 单步步入。
    \item ni: 单步步过。
    \item finish: 执行到当前函数返回。
    \item x/[len][format] 0x123: 查看内存。
    \item p: 输出一个表达式的值。
    \item b: 断点,"b\ *[func/addr]"。
    \item info b/info bl: 列出断点。
    \item del: 删除断点,"del\ id"。
    \item clear: 删除指定位置的断点。"clear *[func/addr]"。
    \item set: 修改数据,"set\ $\$$reg=value","set\ {type}addr=value"。
\end{itemize}

\paragraph*{IDA和pwngdb} pwndbg可以远程使用IDA功能进行调试。
\begin{itemize}
    \item 在IDA中运行pwndbg中的ida\_script.py,IDA会监听本地31377端口。如果在虚拟机中使用pwndbg,在host使用IDA,需要把脚本中127.0.0.1改为0.0.0.0来允许虚拟机连接。
    \item 在gdb中执行config idarpc-host "hostIP",重启gdb即可生效。
\end{itemize}
完成后,pwndbg中可以使用ida的函数名、显示伪代码。如:"b *main",使用$\$$ida("xxx")可以获得名称xxx重定位后的地址。

\subsection{IDA调试器}
IDA也有调试后端,可以调windows32/64bit程序,也可以远程调linux程序。

本地调试器:Local\ Windows\ Debugger,快捷键与x64DBG相同,可以在伪代码上下断点调试,也可以在Debugger->Debugger\ windows->Locals中看局部变量。

远程调试器:IDA的远程调试服务器位于dbgsrv目录,在远程主机上运行后在本地选取连接即可。