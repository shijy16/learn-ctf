\chapter{逆向工程}
\section{逆向工程基础}
主要讲了一些汇编、工具的基本常识,略过。


\section{静态分析}
\subsection{IDA使用入门}
讲IDA怎么用。
数据类型快捷键:
\begin{itemize}
    \item U: 取消一个地方已有数据类型定义。
    \item D: 把一个地方变成数据,一直按会修改数据的长度。
    \item C: 让一个位置变成指令。
    \item A: 让一个位置为起点变成以$\backslash 0$结尾的ASCII字符串。
    \item *: 将一个位置定义为数组。
    \item O: 将一个位置定义为地址偏移。
\end{itemize}

函数操作快捷键:
\begin{itemize}
    \item 删除函数: 选中函数后DELETE。
    \item 定义函数: 选中行后P。
    \item 修改函数参数:函数窗口中选中函数Ctrl+E,反汇编窗口选中函数内部Alt+E。
\end{itemize}

导航操作快捷键:
\begin{itemize}
    \item 后退到上一位置:Esc。
    \item 前进到下一位置:Ctrl+Enter。
    \item 跳转到一个地址:G,然后输入地址/名称。
    \item 跳转到某一区段:Ctrl+S。
\end{itemize}

其他:IDA Python、字符串子窗口、十六进制子窗口。

\subsection{HexRays反编译器入门}
\begin{itemize}
    \item 生成伪代码:F5。
    \item Collapse declaration: 折叠函数。
    \item 修改标识符:在标识符上按N。
    \item 切换常量显示格式:右键。
    \item 修改标识符类型:Y。
    \item 添加结构体类型:Insert或右键,而后IDA会自动识别该类型。也可以添加头文件。
    \item 代码跳过:选中后右键,选择fill\ with\ nops。(需要装\href{https://github.com/L4ys/LazyIDA}{LazyIDA插件})
\end{itemize}

\subsection{IDA和HexRays进阶}
\begin{itemize}
    \item main函数查找: 找\_\_lib\_start\_main和start函数。
    \item FLIRT签名: 函数列表中底色为青色的函数表示因为签名问题识别失败,Shift+F5打开Signature列表,然后按Insert,自动新增签名库。
    \item HesRays函数分析失败:
    \begin{itemize}
        \item call\ analysis\ failed: 找函数调用参数时出错,需要手动修改函数的原型声明,如从'int\ \_\_thiscall'改为'int\ \_\_cdecl'。
        \item sp-analysis\ failed: 优化等级较高时,编译器省略了ebp使用,转而使用rsp引用局部变量,HexRays在跟踪rsp时出错。一般是由参数个数或调用约定出错导致IDA对栈指针变化量计算错误,可以在Option-General中打开Stackpointer分析错误。
        \item 指令分析错误: 有时候代码段中会有一些地方插入了不会到达的混淆,导致之后的指令解析错误,需要定位到错误点把这些地方用nop填充。
    \end{itemize}  
\end{itemize}
